\graphicspath{{chapters/07_conclusions/}}
\chapter{Conclusions}

% The Conclusions and future perspectives section briefly summarizes the key outcome of the project and outlines any future directions (at least 1 page, not more than 2 pages).

This thesis introduces HiCONA which is a Python3 package aimed at providing a flexible, user-friendly and efficient framework for the network analysis of Hi-C data. This package was created because, despite the potential insights that this type of analysis could provide, there are currently no options to perform versatile and comprehensive network analysis on Hi-C data. The main reason for this lack of tools is the nature of the data obtained from Hi-C experiments; though the conversion of this data into graphs is almost trivial, the results are high-order and very dense networks, sometimes too big to even fit into memory, which is rather difficult to work with.

To address this need, the package provides multiple functionalities, which can be divided into network generation, network annotation and network analysis. Network generation includes all of those steps allowing to convert a .cool file, a standard format for the storage of Hi-C data, into multiple chromosome-level networks, with nodes representing genomic regions and edges representing the contacts among them. Network annotation comprehends functions to add, remove and manipulate bin annotations. Network analysis provides both general network analysis algorithms, by interfacing with the \textit{graph-tool} library, as well as custom implementations of more specific algorithms. 

This work focused on network generation and all its sub-steps, those being pixel filtering, genomic distance normalization and network sparsification. Pixel filtering was not found to be effective in reducing pixel number, since only loose thresholds can be applied; still, while pixel filtering is almost irrelevant in terms of sparsification results, it seems to be quite impactful for the consistency of replicates. In regards to genomic distance normalization, a very simple yet stable method was introduced which provides normalized distance values that are within a very consistent range and amenable to optimization via memoization. Network sparsification is the crucial step in dealing with the issue of graph density, allowing to reduce the number of edges while trying to preserve the topological properties of the graph, therefore extracting the backbone of the network. First, a consistent method to determine the sparsification score threshold to use was defined. Then, it was shown that the technique is able to reduce substantially the number of pixels, while maintaining most of the correlation among replicates. Moreover, despite the reduction of the average pixel genomic distance after sparsification, its value remains high enough to suggest that the distribution of the pixel genomic distances continues to cover the range of distances in which promoter-enhancer interactions are expected to be found. Finally, sparsification was found to enrich for pixels whose bins are annotated as functional elements.

From a computational efficiency point of view, it was shown that HiCONA does indeed provide very fast processing with minimal RAM footprint, so that even lower end systems can analyze basically any .cool file. This is achieved through efficient chunking and vectorization of the operations. Another major speed up could be attained using parallelization, which is one of the main objectives for the future of the package. 

HiCONA is still currently in the development stage. While network generation and annotation are basically deployment ready, some work is needed for the network analysis part. The node-labels permutation algorithm is already implemented, though some tweaking is required to extract biologically relevant information from it. Contrast subgraphs extraction is another candidate algorithm for implementation, though it would need some substantial changes to fit Hi-C data. 

The objective is for HiCONA to become a full-fledged package available on all major package indexes (Github, PyPI and Conda), with extensive documentation and tutorials. HiCONA was made to be easy to slot in any already existing pipeline, providing new analysis options, as well as to be easily extended with new functionalities, in order to accommodate for any new need which might arise. Hopefully HiCONA will provide an opportunity to learn more from the still very active research field of chromatin biology.
